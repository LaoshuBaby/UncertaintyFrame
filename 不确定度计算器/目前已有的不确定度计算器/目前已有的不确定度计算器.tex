\PassOptionsToPackage{unicode=true}{hyperref} % options for packages loaded elsewhere
\PassOptionsToPackage{hyphens}{url}
%
\documentclass[]{article}
\usepackage{lmodern}
\usepackage{amssymb,amsmath}
\usepackage{ifxetex,ifluatex}
\usepackage{fixltx2e} % provides \textsubscript
\ifnum 0\ifxetex 1\fi\ifluatex 1\fi=0 % if pdftex
  \usepackage[T1]{fontenc}
  \usepackage[utf8]{inputenc}
  \usepackage{textcomp} % provides euro and other symbols
\else % if luatex or xelatex
  \usepackage{unicode-math}
  \defaultfontfeatures{Ligatures=TeX,Scale=MatchLowercase}
\fi
% use upquote if available, for straight quotes in verbatim environments
\IfFileExists{upquote.sty}{\usepackage{upquote}}{}
% use microtype if available
\IfFileExists{microtype.sty}{%
\usepackage[]{microtype}
\UseMicrotypeSet[protrusion]{basicmath} % disable protrusion for tt fonts
}{}
\IfFileExists{parskip.sty}{%
\usepackage{parskip}
}{% else
\setlength{\parindent}{0pt}
\setlength{\parskip}{6pt plus 2pt minus 1pt}
}
\usepackage{hyperref}
\hypersetup{
            pdfborder={0 0 0},
            breaklinks=true}
\urlstyle{same}  % don't use monospace font for urls
\usepackage{longtable,booktabs}
% Fix footnotes in tables (requires footnote package)
\IfFileExists{footnote.sty}{\usepackage{footnote}\makesavenoteenv{longtable}}{}
\setlength{\emergencystretch}{3em}  % prevent overfull lines
\providecommand{\tightlist}{%
  \setlength{\itemsep}{0pt}\setlength{\parskip}{0pt}}
\setcounter{secnumdepth}{0}
% Redefines (sub)paragraphs to behave more like sections
\ifx\paragraph\undefined\else
\let\oldparagraph\paragraph
\renewcommand{\paragraph}[1]{\oldparagraph{#1}\mbox{}}
\fi
\ifx\subparagraph\undefined\else
\let\oldsubparagraph\subparagraph
\renewcommand{\subparagraph}[1]{\oldsubparagraph{#1}\mbox{}}
\fi

% set default figure placement to htbp
\makeatletter
\def\fps@figure{htbp}
\makeatother


\date{}

\begin{document}

\hypertarget{header-n2828}{%
\section{目前已有的不确定度计算器}\label{header-n2828}}

欢迎随时补充

您可随时进行检索

https://github.com/search?p=1\&q=不确定度\&type=Repositories

\hypertarget{header-n2833}{%
\subsection{GitHub上已有的不确定度计算器}\label{header-n2833}}

\begin{longtable}[]{@{}llllllllllll@{}}
\toprule
地址 & 作者 & 使用语言 & 开发环境 & 协议 & 是否打包可直接运行 &
是否有文档 & 支持的平台 & 最新版本 & 最后更新(yyyy.mm.dd) & 星星数 &
备注\tabularnewline
\midrule
\endhead
https://github.com/zzh1996/Physics-Experiment-Calculator & zzh1996 & C\#
& VC\#.net & 无 & 否 & 否 & 未编译 & 未发布 & 2016.08.02 & \textbf{3} &
大物实验 不确定度计算器\tabularnewline
https://github.com/cutecutecat/Uncertainty-Calculator & cutecutecat &
MATLAB & MATLAB未知版本 & 无 & 否 & 否 & MATLAB & 未发布 & 2017.06.16 &
0 & 不确定度计算器\tabularnewline
https://github.com/Tefx/CUM & Tefx & C、Python & Cython & 无 & 否 & 否 &
未编译 & 未发布 & 2013.01.24 & 0 & 不确定度计算\tabularnewline
https://github.com/Tefx/CUM\emph{Mod}Library & Tefx & Python & Python &
无 & 否 & 否 & 脚本语言 & 未发布 & 2013.01.13 & 0 &
不确定度计算模型库\tabularnewline
https://github.com/lsmind/Physical-experiment & lsmind & Python & Python
& \textbf{LGPL-3.0 license} & \textbf{是} & \textbf{是} & 脚本语言 &
\textbf{v0.9beta} & 2019.08.15 & 0 & 物理不确定度技术\tabularnewline
https://github.com/Robin0218/JLU-physics-experiment- & Robin0218 & C &
纯代码 & 无 & 否 & 否 & 未编译 & 未发布 & 2019.11.9 & 0 &
吉林大学大物实验不确定度计算(A类,合成不确定度)\tabularnewline
https://github.com/LinsongGuo/Physics-Uncertainty-Calculator &
LinsongGuo & HTML、TeX & 纯代码 & 无 & \textbf{是} & \textbf{是} &
脚本语言 & 未发布 & 2019.04.16 & 0 &
大学物理实验不确定度计算器\tabularnewline
https://github.com/JasonGuo98/physics-data-analyze & JasonGuo98 & Python
& Python & 无 & 否 & 否 & 脚本语言 & 未发布 & 2018.10.04 & 0 &
物理实验数据,计算不确定度等\tabularnewline
https://github.com/Anleeos/Physicaldata-error-analysis & Anleeos &
C++、C & Visual Studio 2015及更新版本 & 无 & \textbf{仅Debug} & 否 &
未编译 & 未发布 & 2019.04.24 & 0 &
处理大学物理实验中的不确定度。\tabularnewline
https://github.com/Frostl4/- & Frostl4 & 无 & 无 & \textbf{GPL-3.0} & 否
& 否 & 未编译 & 未发布 & 2019.04.19 & 0 &
大学物理实验不确定度计算\tabularnewline
https://github.com/yuweikong/physics & yuweikong & Python & Python & 无
& 否 & 否 & 脚本语言 & 未发布 & 2019.10.27 & 1 &
大学物理实验求不确定度A,C\tabularnewline
https://github.com/zhc0757/Experiment\emph{Data}Preprocess & zhc0757 &
C++ & Visual Studio & \textbf{GPL-3.0} & \textbf{是} & 否 &
\textbf{X86、X64} & 未发布 & 2018.10.19 & 0 &
用于在大学物理实验中预处理数据:剔除异常数据、计算不确定度\tabularnewline
https://github.com/LoneWolfEric/College\emph{Physics}Experiment\emph{Data}Process
& LoneWolfEric & C++ & VSCode & 无 & \textbf{是} & 否 &
\textbf{可能为X86或X64} & 未发布 & 2018.04.17 & 1 &
用于处理大物实验数据,输入原始数据,自动获得平均值,第一类、第二类不确定度\tabularnewline
& & & & 无 & 否 & 否 & & & & &\tabularnewline
& & & & 无 & 否 & 否 & & & & &\tabularnewline
\bottomrule
\end{longtable}

\hypertarget{header-n3043}{%
\subsection{互联网上已有的不确定度计算器}\label{header-n3043}}

\end{document}
